\documentclass[12pt,oneside,a4paper,fleqn]{ctexart}
\usepackage{geometry}
\usepackage{caption}
\usepackage{graphicx}
\usepackage{setspace}
\usepackage{titletoc}
\usepackage{titlesec}
\usepackage{ctex}

\renewcommand{\abstractname}{\textbf{
    {\zihao{2}题目:$\times \times \times $标题}\\
    \vspace*{1.5em}
\zihao{4}摘\quad 要 \vspace*{0.5em}}}%自定义摘要
\renewcommand{\contentsname}{\zihao{3} \heiti{目\quad 录} \vspace{0.5em}}%自定义目录
\geometry{left=2.1cm,right=1.8cm,top=2.6cm,bottom=1.3cm} % 页边距设置[width=10.5cm,height=2.25cm]
%%%%%%%%%%%%%%%%%%%%%%%%自定义目录样式%%%%%%%%%%%%%%%%%%%%%%%%%%%%%%
\titlecontents{section}[1cm]{\bf \zihao{-4}}{\contentslabel{2.5em}}{}{\titlerule*[0.5pc]{$\cdots$}\contentspage\hspace*{1cm}}
\titlecontents{subsection}[2cm]{\zihao{-4}}{\contentslabel{2.5em}}{}{\titlerule*[0.5pc]{$\cdots$}\contentspage\hspace*{1cm}}
\titlecontents{subsubsection}[3cm]{\zihao{-4}}{\contentslabel{2.5em}}{}{\titlerule*[0.5pc]{$\cdots$}\contentspage\hspace*{1cm}}
%%%%%%%%%%%%%%%%%%%%%%%%自定义section样式%%%%%%%%%%%%%%%%%%%%%%%%%%%%%%

%%%%%%%%%%%%%%%%%%%%%%%%%%%准备结束%%%%%%%%%%%%%%%%%%%%%%%%%%%%%%

\begin{document}
%%%%%%%%%%%%%%%%%%%%%%%%%%%标题页开始%%%%%%%%%%%%%%%%%%%%%%%%%%%
    \begin{titlepage}
        \heiti
        \ 
        \vspace{3em}
        \begin{center}
            \includegraphics{figure/1}\newline
            {\zihao{0}课程设计(论文)}\\
            \vspace{15em}
            \noindent\makebox[70pt][c]{\zihao{-3}课程名称:}\ \ \underline{\makebox[10em]{\zihao{4}$\times \times \times $}}\\
            \vspace{1em}
            \makebox[70pt][s]{\zihao{-3}题目:}\ \ \underline{\makebox[10em]{\zihao{4} $\times \times \times $}}\\
            \vspace{1em}
            \makebox[70pt][s]{\zihao{-3}院(系):}\ \ \underline{\makebox[10em]{\zihao{4} $\times \times \times $工程学院}}\\
            \vspace{1em}
            \makebox[70pt][s]{\zihao{-3}专业班级:}\ \ \underline{\makebox[10em]{\zihao{4}$\times \times \times $}}\\
            \vspace{1em}
            \makebox[70pt][s]{\zihao{-3}姓名:}\ \ \underline{\makebox[10em]{\zihao{4}$\times \times \times $}}\\
            \vspace{1em}
            \makebox[70pt][s]{\zihao{-3}学号:}\ \ \underline{\makebox[10em]{\zihao{4} $\times \times \times $}}\\
            \vspace{1em}
            \makebox[70pt][s]{\zihao{-3}指导教师:}\ \ \underline{\makebox[10em]{\zihao{4}$\times \times \times $}}\\
            \vspace{10em}
            {\zihao{4}2022年1月5号}
        \end{center}
    \end{titlepage}
%%%%%%%%%%%%%%%%%%%%%%%%%标题页结束%%%%%%%%%%%%%%%%%%%%%%%%%%%%%
\newpage
%%%%%%%%%%%%%%%%%%%%%%%%%%摘要页开始%%%%%%%%%%%%%%%%%%%%%%%%%%%%

\begin{abstract}
    \thispagestyle{empty}
    \addcontentsline{toc}{section}{摘要}
    \songti
    \zihao{-4}
    \begin{spacing}{1.25}
        摘要又称概要、内容提要,意思是摘录要点或摘录下来的要点。 [1]  摘要是以提供文献内容梗概为目的,不加评论和补充解释,简明、确切地记述文献重要内容的短文。其基本要素包括研究目的、方法、结果和结论。具体地讲就是研究工作的主要对象和范围,采用的手段和方法,得出的结果和重要的结论,有时也包括具有情报价值的其它重要的信息。
        \newline
        \newline
        \textbf{关键字:}{$\times \times \times $}
    \end{spacing}
    \pagenumbering{Roman}
\end{abstract}

%%%%%%%%%%%%%%%%%%%%%%%%摘要页结束%%%%%%%%%%%%%%%%%%%%%%%%%%%%%%
\newpage
%%%%%%%%%%%%%%%%%%%%%%%%目录页开始%%%%%%%%%%%%%%%%%%%%%%%%%%%%%%
\tableofcontents
\pagenumbering{arabic}




\thispagestyle{empty}
%%%%%%%%%%%%%%%%%%%%%%%%目录页结束%%%%%%%%%%%%%%%%%%%%%%%%%%%%%%
\newpage
%%%%%%%%%%%%%%%%%%%%%%%%%正文开始%%%%%%%%%%%%%%%%%%%%%%%%%%%%%%
\setcounter{page}{1}

\vspace{0.5em} \section{\heiti $\times \times \times $} \vspace{0.5em}


\vspace{0.5em} \subsection{\heiti $\times \times \times $}


\vspace{0.5em} \subsection{\heiti $\times \times \times $}


\vspace{0.5em} \subsection{\heiti $\times \times \times $}


\vspace{0.5em} \section{\heiti $\times \times \times $} \vspace{0.5em}


\vspace{0.5em} \subsection{\heiti $\times \times \times $}


\vspace{0.5em} \subsection{\heiti $\times \times \times $}


\vspace{0.5em} \subsection{\heiti $\times \times \times $}


\vspace{0.5em} \section*{\heiti 参考文献} \vspace{0.5em}
\addcontentsline{toc}{section}{参考文献}

\vspace{0.5em} \section*{\heiti 附录} \vspace{0.5em}
\addcontentsline{toc}{section}{附录}

\vspace{0.5em} \section*{\heiti 致谢} \vspace{0.5em}
\addcontentsline{toc}{section}{致谢}

\end{document}